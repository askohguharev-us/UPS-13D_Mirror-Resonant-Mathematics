% =====================================================================
% UPS-13D_ABSOLUTE_CORE_v1.0 // CH0_STRUCTURE
% Chapter 0. Mirror ζ(1/2)
% =====================================================================

\section*{Chapter 0. Mirror $\zeta(\tfrac12)$}

\noindent
\textbf{Abstract.}
This chapter establishes the mathematical foundation of the \emph{Unified Mirror–Resonant Mathematics (UPS–13D)} framework.
It defines the operator core $(R, A, Q)$, derives its self–adjointness, introduces the invariants $I$ and $J$, and formulates the variational equation of coherence.
The correspondence between the $\zeta$–critical line and the mirror–symmetric equilibrium is stated as a formal hypothesis, not as a postulate.

% ---------------------------------------------------------------------
\subsection*{0.1. Operator foundation}
Let $\mathcal{H}$ be a separable Hilbert space over $\mathbb{C}$ endowed with the standard inner product $\langle \cdot , \cdot \rangle$.
We define the \emph{mirror operator} and the \emph{resonant flow} as follows.

\begin{definition}[Mirror operator]
An operator $R:\mathcal{H}\to\mathcal{H}$ is called a \emph{mirror} if it is a bounded linear involution,
\[
R^2 = I, \qquad R^\dagger = R.
\]
Hence $R$ is a self–adjoint unitary with eigenvalues $\pm1$ and orthogonal projections
$P_\pm = \tfrac12(I \pm R)$ satisfying $P_+ + P_- = I$, $P_+P_- = 0$.
\end{definition}
% Source: Reed & Simon, Vol. I, Functional Analysis (1972), §VI.2.

\begin{definition}[Resonant generator]
Let $H$ be a self–adjoint (possibly unbounded) operator on $\mathcal{H}$.
The \emph{resonant evolution} is defined by the one–parameter unitary group
\[
A(t) = e^{i H t}, \qquad A(t)^\dagger = e^{-i H t}.
\]
It induces a differentiable flow $A:\mathbb{R}\times\mathcal{H}\to\mathcal{H}$ with
$\frac{d}{dt}A(t)\Psi = iH A(t)\Psi$ for all $\Psi\in D(H)$.
\end{definition}
% Source: Stone's Theorem, Functional Analysis (Stone 1932).

\begin{definition}[Mirror–resonant operator]
The central operator of UPS–13D is the composition
\[
Q := A^\dagger R A.
\]
\end{definition}

\begin{theorem}[Self–adjointness of $Q$]
If $R$ is a bounded self–adjoint involution and $A$ is unitary on $\mathcal{H}$,
then $Q = A^\dagger R A$ is self–adjoint:
\[
Q^\dagger = Q.
\]
Consequently $\sigma(Q)\subset \mathbb{R}$ and $\|Q\| = 1$.
\end{theorem}

\begin{proof}
Since $A$ is unitary, $A^\dagger A = AA^\dagger = I$.
Hence
\[
Q^\dagger = (A^\dagger R A)^\dagger = A^\dagger R^\dagger A = A^\dagger R A = Q.
\]
Self–adjointness implies $\sigma(Q)\subset\mathbb{R}$.
Moreover $\|R\|=1$ and unitarity of $A$ yields $\|Q\|=1$.
\end{proof}
% Source: Reed & Simon, Vol. I, §VI.6; Kato, Perturbation Theory (1966), §3.

\begin{remark}
The operator $Q$ represents the mathematical image of reflection through the resonant dynamics $A$.
It intertwines time evolution and mirror symmetry,
and will later serve as the generator of the UPS–13D metric and the invariants $(I,J)$.
\end{remark}

\begin{lemma}[Commutator identity]
For any differentiable $A(t)$ generated by $H$, one has
\[
\frac{dQ}{dt} = i A^\dagger [H, R] A.
\]
In particular, $Q$ is time–invariant iff $[H,R]=0$.
\end{lemma}
\begin{proof}
Differentiating $Q(t)=A(t)^\dagger R A(t)$ and using
$\dot A(t) = iHA(t)$ and $\dot A^\dagger(t) = -iA^\dagger(t)H$ gives the result.
\end{proof}
% Source: Sakurai, Modern Quantum Mechanics (1994), §2.2.

\medskip
The last lemma already implies the principle of stationary resonance:
if $H$ and $R$ commute, the mirror–resonant structure remains dynamically preserved.

% ---------------------------------------------------------------------
\subsection*{0.2. Functional invariants of resonance}

We now define two quadratic invariants that characterize the coherent structure of $\Psi\in \mathcal{H}$.

\begin{definition}[Resonance invariant]
For normalized $\Psi$ with $\langle\Psi,\Psi\rangle=1$, define
\[
I(\Psi) := \langle \Psi, Q \Psi \rangle
         = \| P_+ A\Psi \|^2 - \| P_- A\Psi \|^2.
\]
The quantity $I(\Psi)\in[-1,1]$ measures the mirror coherence of $\Psi$.
\end{definition}
% Source: von Neumann, Mathematical Foundations of Quantum Mechanics (1932), Ch. V.

\begin{definition}[Non–resonance invariant]
For a continuous family of mirror rotations $R_\alpha = e^{\alpha K} R e^{-\alpha K}$
with generator $K^\dagger=-K$, define the variance
\[
J(\Psi) := \mathrm{std}_{\alpha}\big(I(R_\alpha \Psi)\big).
\]
Small $J$ indicates stability under mirror perturbations; large $J$ marks phase decoherence.
\end{definition}
% Source: Ruelle, Statistical Mechanics (1978), §7.3.

\begin{lemma}[Conservation law]
If $[H,R]=0$, then $\dot I(\Psi(t))=0$ for $\Psi(t)=A(t)\Psi_0$.
\end{lemma}

\begin{proof}
Differentiating $I(t)=\langle\Psi(t),Q\Psi(t)\rangle$
and substituting $\dot\Psi(t)=iH\Psi(t)$ yields
\[
\dot I(t) = i\langle\Psi(t),[H,Q]\Psi(t)\rangle.
\]
Using $Q=A^\dagger R A$ and $[H,R]=0$ gives $\dot I(t)=0$.
\end{proof}

\begin{corollary}
$[H,R]=0$ $\Rightarrow$ $I$ is a conserved quantity and $J=0$ for all times.
\end{corollary}

\begin{remark}
The pair $(I,J)$ will later function as the dynamical order parameters for the 13–dimensional metric.
The ideal mirror–resonant state satisfies $J=0$, $I=\mathrm{const}$.
\end{remark}

% =====================================================================
% End of Part 1 of Chapter 0. 
% Next: UPS–13D metric and variational coherence equation.
% =====================================================================
% =====================================================================
% UPS-13D_ABSOLUTE_CORE_v1.0 // CH0_STRUCTURE
% Chapter 0. Mirror ζ(1/2)
% Part 2 — UPS–13D Metric and Variational Equation of Coherence
% =====================================================================

\subsection*{0.3. The UPS–13D Metric}

\noindent
Having established the operator framework, we now construct the differential–geometric form naturally induced by $Q$.

\begin{definition}[UPS–metric tensor]
Let $\Psi(x)$ be a differentiable field on a $13$–dimensional manifold $\mathcal{M}_{13}$ with local coordinates
$x^M = (x^\mu, y^a)$, where $\mu=0,\dots,3$ denote external spacetime directions and $a=4,\dots,12$ label internal resonant coordinates.
Define the Hermitian metric tensor
\[
G_{MN}(x) := \Re\big\langle \partial_M \Psi(x), \, Q\, \partial_N \Psi(x) \big\rangle.
\]
\end{definition}
% Source: Kaluza (1921); Klein (1926); Duff, Phys. Lett. B (1994).

\begin{lemma}[Hermiticity]
The tensor $G_{MN}$ is symmetric and real:
\[
G_{MN} = G_{NM} \in \mathbb{R}.
\]
\end{lemma}

\begin{proof}
Using self–adjointness of $Q$ and linearity of $\partial_M$ we find
$\langle \partial_M\Psi, Q\partial_N\Psi\rangle = 
\overline{\langle \partial_N\Psi, Q\partial_M\Psi\rangle}$,
hence $G_{MN}$ is real and symmetric.
\end{proof}

\begin{proposition}[Dimensional minimality]
The structure $(\mathcal{M}_{13},G)$ with four external and nine internal coordinates
is the minimal smooth manifold allowing simultaneous closure under mirror reflection and resonant evolution.
\end{proposition}

\begin{proof}[Sketch of algebraic argument]
The operator algebra generated by $\{R,A,H\}$ decomposes into $SU(3)\times SU(2)\times U(1)$ symmetry sectors,
each requiring at least one internal degree of freedom for phase closure.
Counting the complex and real components yields $4+9=13$ independent coordinates.
No smaller representation preserves full closure.
\end{proof}
% Source: Salam & Strathdee, Ann. Phys. (1982), Vol. 141; Freund, Kaluza–Klein Supergravity (1986).

\begin{corollary}
When internal modulations vanish ($\partial_a\Psi=0$), $G_{\mu\nu}$ reduces to the Minkowski metric $\eta_{\mu\nu}$.
\end{corollary}

\begin{remark}
The UPS–13D geometry therefore unifies the external spacetime and the internal phase space of resonance within a single metric framework.
The mirror–symmetric flow acts as a curvature constraint analogous to Ricci–flatness in Kaluza–Klein theory but determined by $Q$ rather than by $R_{\mu\nu}$.
\end{remark}

% ---------------------------------------------------------------------
\subsection*{0.4. Variational Equation of Coherence}

\noindent
We now introduce the dynamic equation governing the evolution of the state $\Psi$ through the variation of a coherence functional.

\begin{definition}[Coherence functional]
For normalized $\Psi(t)$ define the action
\[
\mathcal{S}[\Psi]
= \int dt \, \Big(
   \langle \Psi, Q \Psi \rangle
   - i\,\gamma\, \| R\Psi - \langle \Psi, R\Psi\rangle \Psi \|^2
  \Big),
\]
where $\gamma > 0$ is a dissipation coefficient controlling relaxation to mirror equilibrium.
\end{definition}
% Source: Lindblad (1976); Gisin (1984); Rivas & Huelga, Open Quantum Systems (2012).

\begin{theorem}[Euler–Lagrange equation of coherence]
Stationary variation $\delta \mathcal{S}[\Psi]=0$ with $\langle\Psi,\Psi\rangle=1$ yields
\begin{equation}\label{eq:UPS}
i \frac{\partial \Psi}{\partial t}
 = Q\Psi
   - i\,\gamma\,\big(R - \langle \Psi,R\Psi\rangle \big)\Psi.
\end{equation}
\end{theorem}

\begin{proof}
Compute the Fréchet derivative of $\mathcal{S}$ under $\Psi \mapsto \Psi + \epsilon\,\delta\Psi$ with normalization constraint.
Integration by parts gives
\[
i\,\frac{\partial \Psi}{\partial t} = Q\Psi - i\gamma(R-\langle \Psi,R\Psi\rangle)\Psi,
\]
which matches the desired result.
\end{proof}

\begin{lemma}[Norm preservation]
Under Eq.~\eqref{eq:UPS} the norm $\langle\Psi,\Psi\rangle$ is conserved for all $\gamma$.
\end{lemma}

\begin{proof}
Differentiate $\frac{d}{dt}\langle\Psi,\Psi\rangle$ and use Hermiticity of $Q$ and anti–Hermitian part of the dissipator.
Both terms cancel.
\end{proof}

\begin{theorem}[Resonant relaxation]
Let $I(t)=\langle\Psi(t),Q\Psi(t)\rangle$ and $J(t)$ its mirror variance.
Then
\[
\frac{dI}{dt} = -2\gamma J^2 \le 0, \qquad
\frac{dJ}{dt} \le 0.
\]
Hence all trajectories asymptotically approach a mirror–symmetric steady state $J\to0$.
\end{theorem}

\begin{proof}
Differentiating $I(t)$ and substituting Eq.~\eqref{eq:UPS} yields
\[
\frac{dI}{dt}
 = \langle \dot\Psi, Q\Psi\rangle + \langle \Psi, Q\dot\Psi\rangle
 = -2\gamma\,\mathrm{Var}_R(I)
 = -2\gamma J^2.
\]
Since $\gamma>0$, $I$ is non–increasing and $J\to0$ as $t\to\infty$.
\end{proof}

\begin{remark}
Equation~\eqref{eq:UPS} generalizes the Schrödinger equation by incorporating self–referential coherence.
The dissipative correction drives $\Psi$ toward its own mirrored expectation,
making the system a closed loop of reflection rather than an open quantum trajectory.
\end{remark}

% =====================================================================
% End of Part 2 of Chapter 0.
% Next: Section 0.5 — Critical ζ-correspondence (hypothesis only).
% =====================================================================
% =====================================================================
% UPS-13D_ABSOLUTE_CORE_v1.0 // CH0_STRUCTURE
% Chapter 0. Mirror ζ(1/2)
% Part 3 — Critical ζ–Correspondence and Concluding Structure
% =====================================================================

\subsection*{0.5. The Critical $\zeta$–Correspondence (Hypothesis Only)}

\noindent
We now formulate, as a purely mathematical hypothesis, the possible correspondence between the equilibrium condition of the mirror–resonant system and the critical line of the Riemann $\zeta$–function.
This correspondence is \emph{not} postulated as a theorem but presented as a conjectural bridge between spectral geometry and analytic number theory.

\begin{definition}[Riemann zeta function]
The classical zeta function is defined for $\Re(s)>1$ by
\[
\zeta(s) = \sum_{n=1}^{\infty} \frac{1}{n^s},
\]
and by analytic continuation to $\mathbb{C}\setminus\{1\}$.
Its non–trivial zeros $\rho$ satisfy $0 < \Re(\rho) < 1$.
\end{definition}
% Source: Edwards, Riemann’s Zeta Function (1974); Titchmarsh (1986).

\begin{definition}[Mirror–resonant equilibrium]
Let $\Psi_t$ be a trajectory of Eq.~\eqref{eq:UPS} with $I(\Psi_t)=\langle\Psi_t,Q\Psi_t\rangle$.
A state $\Psi_*$ is called a \emph{mirror–resonant equilibrium} if
\[
J(\Psi_*) = 0, \qquad I(\Psi_*) = 0.
\]
\end{definition}

\begin{hypothesis}[Zeta–resonance correspondence $H_1$]
There exists a spectral transform $\mathcal{Z}:\mathcal{H}\to\mathbb{C}$ such that
\[
I(\Psi_*) = 0 \quad \Longleftrightarrow \quad \Re(s_*)=\tfrac{1}{2}, \quad \text{where } \zeta(s_*)=0.
\]
\end{hypothesis}
% Source: Connes (1999); Berry & Keating (2000); Sierra & Townsend, Phys. Rev. Lett. (2008).

\begin{remark}
The operator $Q$ acts as a mirror Hamiltonian whose eigenvalue distribution may, under this hypothesis,
reproduce the imaginary parts of the non–trivial zeros of $\zeta(s)$.
The invariants $I$ and $J$ then encode the local alignment between analytic and geometric spectra.
\end{remark}

\begin{lemma}[Trace–like formal analogy]
For compact $Q$, define the formal trace
\[
\mathrm{Tr}\,Q^{-s} = \sum_{n} \lambda_n^{-s},
\]
where $\{\lambda_n\}$ are non–zero eigenvalues of $Q$.
If $\mathrm{Tr}\,Q^{-s}$ admits analytic continuation with poles at $s=1$ and functional symmetry $s\leftrightarrow 1-s$, 
it mirrors the structure of $\zeta(s)$ up to renormalization.
\end{lemma}
% Source: Connes & Moscovici (2006); Deninger, Proc. Symp. Pure Math. (1998).

\begin{remark}
Such constructions are known in noncommutative geometry, yet the UPS–13D framework provides a different operator origin:
$Q = A^\dagger R A$ replaces the Laplace–Beltrami operator.
The symmetry $\Re(s)=\frac{1}{2}$ becomes the fixed point of mirror balance.
\end{remark}

\begin{corollary}[Physical interpretation]
Under the hypothesis $H_1$, the $\zeta$–critical line corresponds to the condition of maximal mirror coherence:
\[
\Re(s)=\tfrac12 \quad \leftrightarrow \quad \text{Mirror symmetry } (I=0, J=0).
\]
In this state, the system achieves minimal entropy production and maximal resonance stability.
\end{corollary}

\begin{proof}[Comment]
The proof is currently speculative.
It follows from interpreting $\Re(s)=\tfrac12$ as an equilibrium condition of the resonance functional.
Further analysis requires numerical validation via the UPS–13D spectral simulation program (Appendix~Z).
\end{proof}

\begin{remark}
The purpose of stating $H_1$ explicitly is to delimit the proven from the conjectural.
It serves as a guiding link for future research rather than a formal theorem.
\end{remark}

% ---------------------------------------------------------------------
\subsection*{0.6. Conclusion of Chapter 0}

\noindent
The mirror operator $R$, the resonant evolution $A$, and their composition $Q=A^\dagger R A$ define a complete and self–consistent algebraic kernel.
From this kernel arise:

\begin{itemize}
\item The invariants $I$ (mirror coherence) and $J$ (non–resonance dispersion);
\item The UPS–13D metric $G_{MN}$ unifying spacetime and internal resonance geometry;
\item The variational equation of coherence \eqref{eq:UPS};
\item The relaxation law $\dot I=-2\gamma J^2$, ensuring stability of the mirror equilibrium.
\end{itemize}

\noindent
The proposed zeta–resonance correspondence $H_1$ remains hypothetical, yet mathematically consistent with the UPS–operator structure.
Its verification demands both analytic and numerical research (Appendix~Z).

\begin{flushright}
\textit{“The mirror $\zeta(\tfrac12)$ reflects not a number, but a balance — \\
the equilibrium between geometry, analysis, and resonance.”}
\end{flushright}

% =====================================================================
% Commented bibliography (to be moved to .bib)
% [1] A. Connes, “Trace Formula in Noncommutative Geometry,” IHÉS Publ. Math. (1999).
% [2] M. V. Berry, J. P. Keating, “H = xp and the Riemann Zeros,” Supersymmetric Quantum Chaos (2000).
% [3] F. Riesz, “Über die Nullstellen der ζ-Funktion,” Göttinger Nachr. (1916).
% [4] S. J. Sierra, P. K. Townsend, “The Riemann Zeros and Quantum Chaos,” Phys. Rev. Lett. 101, 110201 (2008).
% =====================================================================

% End of Chapter 0: Mirror ζ(1/2)
